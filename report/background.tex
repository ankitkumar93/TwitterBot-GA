\section{Background}
\paragraph{}
Twitter-bots are small software programs that are designed to mimic human tweets. Anyone can create bots, though it usually requires some programming knowledge. Some bots are designed to reply to other users when they detect specific keywords. Others may be designed to randomly tweet preset phrases such as proverbs. Or, if the bot is designed to emulate a popular person (celebrity, historic icon, anime character etc.) their famous quotes may be what makes up the content of the bot's tweets \cite{twitterBot}.
\paragraph{}
These bots can be generated using a variety of APIs, one of which is Tracery \cite{tracery}. Tracery is a super-simple tool and language to generate text (developed by Kate Compton, a.k.a GalaxyKate). It's been used by middle school students, humanities professors, indie game developers, professional bot makers, and lots of regular people too. Although not it's primary use, Tracery provides APIs to integrate the content generated by it, with a Twitter bot.
\paragraph{}
In the world of machine learning, a genetic algorithm (GA) is a metaheuristic inspired by the process of natural selection. Genetic algorithms belong to the larger class of evolutionary algorithms (EA) for machine learning. They are commonly used to generate high-quality solutions for optimization and search problems, by relying on bio-inspired operators such as mutation, crossover and selection\cite{geneticAlgorithm}. The idea is to traverse through a problem's search space, narrowing it down, using a genetic algorithm with an appropriate evaluation function to determine which solutions are "better" than others. We start off with an initial population of solutions which are evaluated using the fitness function. The part of the population that produces a fitness value that is higher than some threshold fitness value is retained, while the ones with a lower fitness value are discarded. A part of the retained population is then used to produce off springs which posses characteristics that are similar to their parents, but modified in some manner. Cross over and selection operations produce generations of the population, where each generation is guaranteed to be at least as fit as the previous generation. However, the solutions found using these two operations only, have a tendency to get stuck in local maxima (or minima depending on the nature of the problem being solved). To avoid this problem, genetic algorithms use another operator called mutation, which introduces random changes in the population in hopes that it will move the solution out of possible local maximas (or minimas).