\documentclass[11pt]{article}
\usepackage[margin=1in]{geometry}
\title{Twitter Bot with Genetic Algorithm for Tweet Optimization}
\author{Ankit Kumar(akumar18), Anand Purohit(apurohi)}
\date{}
\begin{document}
    \maketitle

    \section{Project}
        \subsection{Introduction}
        \paragraph{}
        Twitter bots, software robots which generate tweets and push them out to twitter, are very popular these days. These bots are programmed to generate tweets using different techniques and for different purposes. From a student, to a large company, everyone is interested in these bots since generating good tweets is a really interesting and complicated task. Tweet generation takes into account not only the grammar of a language, but also the context, and thus it makes the task as quite complicated.
        \paragraph{}
        Tweet generation can be broadly divided into two major categories, those that use Grammars, and those that use Machine Learning. While grammars tend to be easier to comprehend, coming up with a grammar that has all the rules needed for meaningful tweet generation, is challenging. On the other hand, machine learning seems like an easier way to bypass all this thinking, and is able to generate much better tweets. However, implementing machine learning is quite a complex task, and training the process itself requires a lot of effort on the developer's end.
        \paragraph{}
        This presents us with a few problems. Could we use grammars to generate better tweets in an easier manner? Could we overcome the overhead of machine learning? Could we come up with something different that enables us to generate better tweets, and without much overhead? We need a way to design twitter bots than can utilize the simplicity of representation of a grammar and the power of machine learning, in order to generate tweets that follow the grammar of a language and also generate meaningful and relevant tweets.
        \paragraph{}
        Any such bot would provide a more simple way to generate well designed tweets, that are meaningful and relevant.

        \subsection{Background}
        \paragraph{}
        Twitter-bots are small software programs that are designed to mimic human tweets. Anyone can create bots, though it usually requires programming knowledge. Some bots reply to other users when they detect specific keywords. Others may randomly tweet preset phrases such as proverbs. Or if the bot is designed to emulate a popular person (celebrity, historic icon, anime character etc.) their popular phrases may be tweeted \cite{twitterBot}.
        \paragraph{}
        These bots can be generated using a variety of APIs, one of which is Tracery. Tracery is a super-simple tool and language to generate text, by Kate Compton, a.k.a GalaxyKate. It's been used by middle school students, humanities professors, indie game developers, professional bot makers, and lots of regular people, too \cite{tracery}. Although not its primary use, Tracery provides APIs to integrate the content generated by it, with a Twitter bot.
        \paragraph{}
        In the world of machine learning, a genetic algorithm (GA) is a metaheuristic inspired by the process of natural selection that belongs to the larger class of evolutionary algorithms (EA). Genetic algorithms are commonly used to generate high-quality solutions to optimization and search problems by relying on bio-inspired operators such as mutation, crossover and selection \cite{geneticAlgorithm}. We can traverse through a problem's search space, narrowing it down, using a genetic algorithm with an appropriate evaluation function to determine the fitness value of solutions. The idea is to start off with an initial population of solutions, that are evaluated using the fitness function. The part of the population that produces a fitness value that is higher than some threshold fitness value is retained, while the ones with a lower fitness value are discarded. A part of the retained population is then to produce offsprings which possess characteristics that are similar to their parents, but mutated in some manner.

        \subsection{Approach}
        \paragraph{}
        Our aim here, is to design a twitter bot, which generates tweets using a simple grammar, and then evolves that tweet by using genetic algorithm and a repository of old tweets. This broadly divides our approach into 2 parts, i.e. Tweet generation using grammars, and Tweet Evolution using Genetic Algorithms. 
        \paragraph{}
        Tweet generation involves using grammars with tools such as Tracery, which adhere to the linguistic rules. This process is responsible for generating well structured, meaningful tweets.
        \paragraph{}
        Tweet Evolution involves using Genetic Algorithm to evolve the tweet generated in the previous phase and output an optimized tweet. This process would make sure that the tweet is relevant to the context and has a clear underlying meaning. For this part, we will be coming up with 3 different fitness functions, based on the similarity of tweets to old tweets from a repository, number of likes and number of re-tweets of those old tweets.
        \paragraph{}
        Due to the enormous number of domains and possibilities, we would limit ourselves to a particular domain only, for out project.

    \section{Work Plan}
        \subsection{Tools and Deliverables}
        \begin{itemize}
            \item \textbf{Tools:} Tracery, Twitter Bot API, Genetic Algorithms, Repository of Old Tweets 
            \item \textbf{Deliverables:} Twitter Bot (Deployed), Code, Documentation, Project Report
        \end{itemize}

        \subsection{Time line and work division}
        We have tried to divide the work evenly, however if needed , we would re-balance the work load accordingly.
        \begin{itemize}
            \item Weeks 1-2: Getting acquainted with the basic tools, algorithms and technologies needed. We would be working individually to learn Tracery, genetic algorithms, Twitter API, etc. We will synchronize a couple of times a week, to coordinate and share what we learned.
            \item Weeks 3-4:  Using the knowledge we gathered to design a simple tweet generator for our bot. We would be working on generating simple tweets using grammars. For this part, we would be working together (pair-programming), instead of working individually.
            \item Weeks 5-6: Developing our genetic algorithm and then combining it with out tweet generator. Here, both of us would first sit together to decide what the metrics of evaluation would be and then we will individually use these metrics to work on different fitness functions. After that we would sit together again to combine our thoughts and ideas and then we would implement the genetic algorithm. Further, one person would work on using the twitter bot API to post out tweets online, while the other would work on interfacing the tweet generator with the genetic algorithm.
            \item Weeks 7-8: Implementing a reply feature for our bot and evaluating its replies (stretch goal). We would be working together to improve our entire tweet generation process. Further we would be working on a reply feature for which, one person would be creating APIs for using another bot to post some tweets to our bot, while the other would work on implementing a reply feature for out main bot.
        \end{itemize}

        \subsection{Evaluation}
        We have 4 basic milestones and a stretch goal in mind, and hence believe each of these can work as a grade level
        \begin{itemize}
            \item D: Simple tweet generator, using Tracery. This tweet generator may be somewhat random, incorrect and its tweets might not have a clear meaning/relevance to the context.
            \item C: Tweet generator that adheres properly to the linguistic rules. The meaning part, might be absent here as well.
            \item B: Tweet generator that not only adheres to the English grammar, but also generates meaningful tweets.
            \item A: Tweet generator that uses the genetic algorithm to improve its tweets. The tweets here must also have some relevance to the context, on top of what is expected for grade 'B'.
            \item A+: Twitter bot for grade 'A' plus a reply feature that enables the bot to reply to others' tweets on its wall.
        \end{itemize}
        
        \begin{thebibliography}{9}
            \bibitem{twitterBot}
            Akimoto, A. (2011). Japan the Twitter nation. The Japan Times. Retrieved from \\\texttt{http://www.japantimes.co.jp/life/2011/05/18/digital/japan-the-twitter-nation/}
            \bibitem{tracery}
            Taken from \\\texttt{http://www.tracery.io/}
            \bibitem{geneticAlgorithm}
            Mitchell, Melanie (1996). An Introduction to Genetic Algorithms. Cambridge, MA: MIT Press. ISBN 9780585030944.
        \end{thebibliography}
\end{document}